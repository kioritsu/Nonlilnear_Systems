\documentclass{jsarticle}
\usepackage{amssymb}
\usepackage{graphicx}
\usepackage[dvipdfmx]{color}
\usepackage{here}
\usepackage{tabularx}
\usepackage{amsmath}
\usepackage{amsthm}
\usepackage{url}
\usepackage[hang,small,bf]{caption}
\usepackage[subrefformat=parens]{subcaption}
\usepackage{tikz}
\usepackage{siunitx}
\usepackage{bm}
\usepackage{ascmac}
\usepackage{tasks}
\usepackage[top=15truemm,bottom=20truemm,left=20truemm,right=20truemm]{geometry}
% \usepackage{fancybx}
\usetikzlibrary{shapes.geometric}
\usetikzlibrary {shapes.misc}
\usetikzlibrary{positioning}
\captionsetup{compatibility=false}


\begin{document}

\settasks{counter-format=(\arabic*)} %「=」の後ろで箇条書きの記号を変更

\section*{Ex6.7}
伝達関数$(b_0s + b_1)/(s^2 + a_1s + a_2)$が厳密に誠実であるとき、かつその時に限り、\\
すべての係数が正であり、かつ$b_1 <a_1b_0$であることを示せ。

\vspace{1cm}
伝達関数$G(s) = (b_0s + b_1)/(s^2 + a_1s + a_2)$が厳密に誠実であること、つまり
\begin{itemize}
  \item $G(s)$の極の実部が負である
  \item $Re[G(jw)]>0$であること
  \item $G(\infty) = 0 \;\text{and}\;\lim_{w\rightarrow \infty}w^2Re[G(jw)]>0$
\end{itemize}
の条件について考える。

$G(s)$の極は$\frac{-a_1 \pm\sqrt{a_1^2-4a_2}}{2}$となるため、
$G(s)$の極の実部が負であるとき、かつその時に限り、$a_1,a_2 > 0$である。
\begin{align*}
  Re[G(jw)] = Re\left[\frac{b_0jw + b_1}{-w^2 + a_1jw + a_2}\right]=\frac{b_1a_2 + (b_0a_1 - b_1)w^2}{(a_2-w^2)^2+a_1^2w^2}
\end{align*}
ここで、先ほどの条件より、$a_1,a_2 > 0$であるとする。\\
$Re[G(jw)]>0$であるとき、かつその時に限り、$b_1>0\;,\;b_0a_1 \geq b_1$である。
$G(\infty) = 0$であるため、
\begin{align*}
  \lim_{w\rightarrow \infty}w^2Re[G(jw)]= b_0a_1-b_1 
\end{align*}
よって、$\lim_{w\rightarrow \infty}w^2Re[G(jw)]>0$であるとき、かつその時に限り、$b_0a_1 > b_1$である。

そのため、
伝達関数$(b_0s + b_1)/(s^2 + a_1s + a_2)$が厳密に誠実であるとき、かつその時に限り、\\
すべての係数が正であり、かつ$b_1 <a_1b_0$である。




\newpage

\section*{Ex6.19}
(6.21)(6.22) の形式の 2つの時不変動的システムのフィードバック接続を考える。 
\begin{align*}
  \dot x_i = f_i(x_i,e_i) \tag{6.21}\\
  y_i = h_i(x_i,e_i) \tag{6.22}
\end{align*}
両方のフィードバック成分がゼロ状態可観測であり、以下を満たす正定値蓄積関数が存在すると仮定する。
\begin{align*}
  e^\top_i y_i \geq \dot V_i + e_i^\top \varphi_i(e_i)+y_i^\top \rho_i(y_i) \;,\;\text{for}\;i=1,2
\end{align*}
$u = 0$ の閉ループ システム (6.24) は、
\begin{align*}
  v^\top [\rho_1(v) + \varphi_2(v)]>0 \;\text{and}\; v^\top [\rho_2(v)-\varphi_1(-v)]>0\;,\;\forall v \neq 0
\end{align*}
の場合、原点が漸近安定であることを示せ。

また、どのような条件を追加すれば、原点が大域的漸近安定になるか?
\vspace{1cm}
$u=0$のとき、$e_1 = -y_2\;,\;e_2 = y_1$である。\\
閉ループ系のリアプノフ候補関数$V = V_1 + V_2$とする。
$V_1,V_2$が正定値関数であるため、$V$も正定値関数である。
\begin{align*}
  \dot V &\leq e^\top_1 y_1-e_1^\top \varphi_1(e_1)-y_1^\top \rho_1(y_1)+e^\top_2 y_2-e_2^\top \varphi_2(e_2)-y_2^\top \rho_2(y_2)\\
  & = -y_1^\top \{\varphi_2(y_1)+\rho(y_1)\} + y_2^\top \{\rho_2(y_2)-\varphi_1(-y_2)\} \leq 0
\end{align*}
$\dot V = 0\Rightarrow y_1^\top \{\varphi_2(y_1)+\rho(y_1)\}=0 \text{and} y_2^\top \{\rho_2(y_2)-\varphi_1(-y_2)\} =0 \Rightarrow y_1 =0 \text{and} y_2=0$

ここで、
$y_1=0 \Rightarrow x_1=0$,$y_2 = 0\Rightarrow x_2 = 0$となるため、原点が漸近安定である。

また、原点が大域的漸近安定にとなるには、蓄積関数$V_1,V_2$が放射状非有界である必要がある。

\end{document}