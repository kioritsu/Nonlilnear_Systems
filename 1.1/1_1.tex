\documentclass{jsarticle}
\usepackage{amssymb}
\usepackage{graphicx}
\usepackage[dvipdfmx]{color}
\usepackage{here}
\usepackage{tabularx}
\usepackage{amsmath}
\usepackage{amsthm}
\usepackage{url}
\usepackage[hang,small,bf]{caption}
\usepackage[subrefformat=parens]{subcaption}
\usepackage{tikz}
\usepackage{siunitx}
\usepackage{bm}
\usepackage{ascmac}
\usepackage[top=15truemm,bottom=20truemm,left=20truemm,right=20truemm]{geometry}
\usetikzlibrary{shapes.geometric}
\usetikzlibrary {shapes.misc}
\usetikzlibrary{positioning}
\captionsetup{compatibility=false}

\begin{document}
4/5

\section*{Chapter Introduction}
    エンジニアが電気回路、機械システム、制御システム等の工学分野の
    非線形動的システムを解析・設計する場合、幅広い非線形解析のツールを学ぶ必要がある。
    本書では、これらのツールの一部を紹介する。\\
    -リアプノフの方法に着目し、
    非線形システムの安定性解析のためのツールを紹介\\
    -入出力および受動性の観点から、
    フィードバックシステムの安定性に着目\\
    -記述関数法を含む、「自由」振動の検出と分析のためのツールを紹介\\
    -平均化と特異摂動を含む、摂動理論の漸近ツールを紹介\\
    -線形化、ゲイン スケジューリング、積分制御、フィードバック線形化、
    スライディングモード制御、リアプノフ再設計、バックステッピング、
    受動性ベースの制御、高ゲインオブザーバー
    などの非線形フィードバック制御ツールを紹介\\

\section*{1.1非線形モデルと非線形現象}
    有限個($n$個)の連立??一階常微分方程式によってモデル化される
    力学系を扱う。
    \begin{align*}
        \dot x_1 &= f_1(t,x_1,\dots,x_n,u_1,\dots,u_p) \\
        \dot x_2 &= f_2(t,x_1,\dots,x_n,u_1,\dots,u_p) \\
        & \vdots\\
        \dot x_n &= f_n(t,x_1,\dots,x_n,u_1,\dots,u_p) 
    \end{align*}
    ここで、\\
    $\dot x_i$:時間変数$t$についての$x_i$の導関数\\
    $u_1,u_2,\dots, u_p$:入力変数\\
    $x_1,x_2,\dots,x_n$:状態変数、
    力学系の過去の記憶(軌道)を表す\\
    以下のベクトル表記を用いて、$n$次元1階ベクトル微分方程(1.1)として上の式を書き換える。
    \begin{equation*}
        x = \left[
        \begin{matrix}
            x_1 \\ x_2 \\ \vdots \\ \vdots \\ x_n \\
        \end{matrix}    
        \right]\;,\;
        u = \left[
        \begin{matrix}
            u_1 \\ u_2 \\ \vdots \\ u_p
        \end{matrix}    
        \right]\;,\;
        f(t,x,u) = \left[
        \begin{matrix}
            f_1(t,x,u) \\ f_2(t,x,u) \\ \vdots \\ \vdots \\ f_n(t,x,u)
        \end{matrix} 
        \right]
    \end{equation*}
    % n 1 階微分方程式を 1 つの n 次元 1 階ベクトル微分方程式とする
    \begin{equation}
        \dot x = f(t,x,u) \tag{1.1}
    \end{equation}
    (1.1)を状態方程式と呼び、$x$を状態、$u$を入力とする。\\
    別の方程式 
    \begin{equation}
        y = h(t,x,u) \tag{1.2}
    \end{equation}
    が式 (1.1) に関連付けられ、
    力学系の解析で特に重要な変数である 
    $q$次元の出力ベクトル $y$ が定義できる。
    (これは、物理的に測定できる変数、または指定された方法で動作する必要がある変数として用いられることもある)。 
    (1.2) を出力方程式と呼び、方程式 (1.1) と (1.2) をまとめて状態空間モデル、または状態モデルと呼ぶ。 
    有限次元の物理システムの数学的モデルは、常に状態モデルの形式で提供されるとは限らない。 
    しかし、多くの場合、状態変数を慎重に選択することで、物理システムをこの形式でモデル化できる。 
    この章で後述する例と演習では、状態モデルの多用途性を示す。
    \newpage
    この本での分析の大部分は、入力$u$が明示的に存在しない状態の方程式、
    つまりいわゆる強制されない(自由?)状態方程式[unforced state equation](1.3)を何度も扱う。
    \begin{equation}
        \dot x = f(t,x) \tag{1.3}
    \end{equation}
    強制されない状態方程式は、システムへの入力がゼロである必要はない。 
    入力は、与えられた時間の関数$u= \gamma(t)$、与えられた状態のフィードバック関数$u= \gamma(x)$、
    もしくは、その両方の$u = \gamma ( t , x )$として指定されている可能性がある。
    式(1.1) に$u = \gamma$ を代入すると、$u$ が削除され、強制されない状態方程式(1.3)が得られる。
    
    関数$f$が明示的に時間$t$に依存しない場合、(1.3)の特殊なケースとなる。 
    つまり、
    \begin{equation}
        \dot x = f(x) \tag{1.4}
    \end{equation}
    の場合、システムは自律的[autonomous]または時不変[time invariant]であると呼ぶ。 
    自律システムは、
    $t$ から $\tau = t -a$ への時間変数の変化によって、
    状態方程式(1.4)の右辺は変化しないため、時間原点の変化に対して不変である。
    システムが自律的でない場合は、非自律的[nonautonomous]または時変[time varying]と呼ばれる。 
    
    状態方程式を扱う上で、平衡点の概念が重要。 
    状態空間の点$x = x^*$は、システムの状態が$x^*$で始まるときはいつでも、
    その先の時間ずっとその点$x^*$にとどまるという特性を持っている場合、(1.3)の平衡点であるとよぶ。 
    自律システム (1.4) の場合、
    平衡点は$f(x)=0$の実根と一致する。
    平衡点は孤立し得る、つまり、その付近に他の平衡点がないか、
    平衡点の連続体が存在する可能性があります。
    % ここ良く分からん

    線形システムの場合、状態モデル (1.1)-(1.2)は次の形式を取る。
    \begin{align}
        \dot x = & A(t)x+B(t)u \\
        y =& C(t)x + D(t)u
    \end{align}
    読者の皆様は、重ね合わせの原理を基礎とした線形システムの強力な解析ツールに精通していることと思います。
    線形システムから非線形システムに移行すると、より困難な状況に直面することになる。
    重ね合わせの原理はもはや通用せず、解析ツールにはより高度な数学が必要となる。
    線形システムには強力なツールがあるため、
    非線形システムを解析する最初のステップは、
    通常、可能であれば、ある公称動作点に関して線形化し、得られた線形モデルを解析することです。
    これはエンジニアリングの世界では一般的な手法であり、有用なものである。
    非線形システムの挙動を知るために、可能な限り線形化を行うべきであることに疑問の余地はない。
    しかし、線形化だけでは十分ではなく、非線形システムの解析ツールを開発する必要があります。
    線形化には、2つの基本的な限界があります。
    第一に、線形化は動作点の近傍での近似であるため、
    その点の近傍での非線形システムの「局所的」な挙動しか予測できない。
    動作点から遠く離れた「非局所的」な挙動は予測できず、
    状態空間全体にわたる「大局的」な挙動も予測できない。
    第二に、非線形システムのダイナミクスは、線形システムのダイナミクスよりもはるかに豊かである。
    非線形性の存在下でしか起こらない「本質的な非線形現象」があり、線形モデルで記述・予測することはできない。
    以下に、本質的な非線形現象の例を挙げる:

    \begin{itemize}
        \item 有限発散時間:不安定な線形システムの状態は、時間が無限大に近づくにつれて無限大になるが、
        非線形システムの状態は、有限時間で無限大になることができる。第3章で紹介する予定。
        \item 複数の孤立均衡:線形システムは、孤立平衡点を1つだけ持つことができる。
        したがって、初期状態に関係なくシステムの状態を引きつける定常状態の動作点を1つだけ持つことができる。
        一方、非線形システムは、1つ以上の孤立平衡点を持つことができる。
        また、システムの初期状態に応じて、いくつかの定常動作点のうちの1つに収束することがある。
        2章で2次自律系を検討する際に紹介予定。
        \item リミットサイクル:線形時不変システムが振動するためには、虚軸上に一対の固有値を持つ必要がある。
        これは、摂動がある場合に維持することがほとんど不可能な非ロバスト条件。
        仮に振動するとしても、振動の振幅は初期状態に依存することになる。
        現実の世界では、安定した振動は非線形システムによって生み出される。
        初期状態に関係なく、一定の振幅と周波数の振動を起こすことができる非線形システムが存在する。
        このような振動はリミットサイクルとして知られている。
        2章で2次自律系を検討する際に紹介予定
        \item サブハーモニック、ハーモニック、ほぼ周期的な振動:
        周期的な入力の下で安定した線形システムは、同じ周波数の出力を生成する。
        周期的な入力の下で非線形システムは、入力周波数のサブマルチプル(約数)または
        マルチプル(倍数)の周波数で振動することがある。
        また、ほぼ周期的な振動を発生させることもある。
        (例、互いに倍数でない周波数を持つ周期的な振動の和)
        \item カオス:非線形システムは、平衡でもなく、周期的振動でもなく、ほぼ周期的な振動でもない、
        より複雑な定常状態挙動を持つことがある。このような挙動は通常、カオスと呼ばれる。
        このようなカオス的な運動の中には、システムの決定論的な性質にもかかわらず、ランダム性を示すものがある。 
        \item 複数のふるまい:同じ非線形システムで2つ以上のふるまいを示すことは珍しくない。
        非強制系は1つ以上のリミットサイクルを持つことがある。
        周期的な加振を行う強制系は、入力の振幅と周波数に応じて、
        サブハーモニック、ハーモニック、ほぼ周期的な振動のふるまいを示すことがある。
        加振の振幅や周波数を滑らかに変化させると、ふるまいが不連続に変化することもある。
\end{itemize}
強制振動、カオス、分岐、その他の重要なトピックについては、[70]、[74]、[187]、[207]を参照。






\end{document}